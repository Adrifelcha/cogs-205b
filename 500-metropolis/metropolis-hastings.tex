\documentclass[t,10pt]{beamer}

\usepackage{beamerthemeshadow}
\usepackage{ulem}
\usepackage{listings}
\usepackage{color} %red, green, blue, yellow, cyan, magenta, black, white
\usepackage{hyperref}


\title{{\color{black}COGS 205B: Computational Lab Skills for Cognitive Scientists}}

\definecolor{amber}{rgb} {0.99, 0.75, 0.20}
\definecolor{violet}{rgb}{0.55, 0.20, 0.90}
\definecolor{jade}{rgb}  {0.00, 0.66, 0.42}
\definecolor{blue}{rgb}  {0.00, 0.39, 0.64}
\definecolor{gold}{rgb}  {1.00, 0.82, 0.00}
\definecolor{lightgrey}{rgb}  {0.85, 0.85, 0.85}
\definecolor{offwhite}{rgb}   {0.965, 0.965, 0.965}

\definecolor{mygreen}{RGB}{28,172,0} % color values Red, Green, Blue
\definecolor{mylilas}{RGB}{170,55,241}
\definecolor{darkblue}{RGB}{0,0,121}
\definecolor{darkgray}{RGB}{48,48,128}
\definecolor{gold}{RGB}{255,127,0}

\sbox0{\small\ttfamily A}
\edef\mybasewidth{\the\wd0 }
\lstset{language=Matlab,%
    breaklines=true,%
    basicstyle=\small\ttfamily,% print whole listing small
    columns=fixed,basewidth=\mybasewidth,
    morekeywords={matlab2tikz},
    keywordstyle=\color{blue},%8
    morekeywords=[2]{1}, keywordstyle=[2]{\color{black}},
    identifierstyle=\color{black},%
    stringstyle=\color{mylilas},
    commentstyle=\color{mygreen},%
    showstringspaces=false,%without this there will be a symbol in the places where there is a space
    numbers=left,%
    numberstyle={\tiny \color{black}},% size of the numbers
    numbersep=9pt, % this defines how far the numbers are from the text
    flexiblecolumns=true,%
    emph=[1]{for,end,break},emphstyle=[1]\color{red}, %some words to emphasise
    %emph=[2]{word1,word2}, emphstyle=[2]{style},    
}

\usepackage{array}
\newcolumntype{L}[1]{>{\raggedright\let\newline\\\arraybackslash\hspace{0pt}}m{#1}}
\newcolumntype{C}[1]{>{\centering\let\newline\\\arraybackslash\hspace{0pt}}m{#1}}
\newcolumntype{R}[1]{>{\raggedleft\let\newline\\\arraybackslash\hspace{0pt}}m{#1}}

%\usepackage[framed,numbered]{mcode}
%\usepackage{verbatim}
%\usepackage{url,textcomp}

\newcommand{\red}[1]{{\color{red}#1}}
\newcommand{\instr}[1]{{\color{darkblue}#1}}
\renewcommand{\emph}[1]{{\it\color{gold}#1}}
\newcommand{\U}{{\mathcal{U}}}
\newcommand{\N}{{\mathcal{N}}}
\newcommand{\E}{{\mathcal{E}}}
\newcommand{\D}{{\mathcal{D}}}
\newcommand{\W}{{\mathcal{W}}}
\newcommand{\B}{{\mathcal{B}}}
\newcommand{\M}{{\mathcal{M}}}

\newcommand{\mcode}[1]{{\color{darkgray}\texttt{#1}}}

\newtheorem{exercise}{Exercise}
\newenvironment<>{exercise}[1][]{%
  \setbeamercolor{block title}{fg=white,bg=blue}%
  \begin{block}{Exercise: #1}#2}{\end{block}}
  
\newtheorem{matlabcode}{Code}
\newenvironment<>{matlabcode}[1][]{%
  \setbeamercolor{block title}{fg=white,bg=gray!75!black}%
  \begin{block}{Code: #1}\begin{lstlisting}#2\end{lstlisting}}{\end{block}}

\newtheorem{axiom}{Axiom}
\newenvironment<>{axiom}[1][]{%
  \setbeamercolor{block title}{fg=white,bg=gold!75!white}%
  \begin{block}{Axiom: #1}#2}{\end{block}}

\newcommand{\LectureTitlePage}{%
    \setcounter{framenumber}{0}
    \global\def\inserttitle{{\color{black}Part \insertlecturenumber: \insertlecture}}
    \global\def\insertshorttitle{{\color{black}Part \insertlecturenumber: \insertlecture}}
    \global\def\insertdate{}
    \global\def\insertauthor{}
    %\global\def\insertshortdate{\lecturedate}
  \frame{\titlepage} 
}

\usepackage{amsmath}
\usepackage{graphicx}

% Theme modifications
\usetheme[sectionpage=none,progressbar=foot]{metropolis}
\addtobeamertemplate{frametitle}{}{\vspace{1em}}
\renewcommand{\emph}[1]{{{\color{red}#1}}}

\setbeamercolor{block title}{use=structure,fg=black,bg=black!15!white}
\setbeamercolor{block body}{use=structure,fg=black,bg=black!10!white}

\setbeamercolor{background canvas}{bg=white}
 
\author{Joachim Vandekerckhove}
\date{\the\year} 


\begin{document}

\frame{\titlepage} 

\setcounter{lecture}{7}

\lecture{Metropolis-Hastings sampling}

\LectureTitlePage
\frame{\frametitle{Metropolis-Hastings sampling}
\begin{itemize}
 \item A widely applicable algorithm for numerical integration is the \red{Metropolis-Hastings} (MH) algorithm\pause
 \item In the MH algorithm, we will randomly generate \red{candidate samples} from some simple
 distribution, and then decide to accept or reject the candidate\pause
 \item MH algorithms need some customization and fine-tuning to be most efficient\pause
 \item One intuition for the MH algorithm is that it is a rejection sampler with an adaptive envelope
\end{itemize}
}

\frame{\frametitle{Metropolis-Hastings sampler}
Given a function $f(\theta) \propto p(\theta|D)$, and a symmetric
\red{candidate generating distribution} $Q(a|b)=Q(b|a)$, a
Metropolis-Hastings sampling algorithm proceeds as follows:
\begin{itemize}
 \item[1] \instr{Set} $i \leftarrow 1$ and \instr{choose} $R \leftarrow 1000$
 \item[2] \instr{Choose}, arbitrarily, $\theta^{(0)}$
 \item[3] \instr{Draw} a randomly selected $\theta^{c}$ from $Q\left(\theta|\theta^{(i-1)}\right)$
 \item[4] \instr{Compute} the acceptance ratio $\alpha = \frac{f\left(\theta^{c}\right)}{f\left(\theta^{(i-1)}\right)}
 = \frac{p\left(\theta^{c}|D\right)}{p\left(\theta^{(i-1)}|D\right)}$
 \item[5] \instr{Draw} a randomly selected $u$ from $U(0,1)$. \instr{If} $\alpha > u$, 
          \instr{set} $\theta^{(i)} \leftarrow \theta^{c}$, 
          \instr{otherwise set} $\theta^{(i)} \leftarrow \theta^{(i-1)}$
 \item[6] \instr{Set} $i \leftarrow i + 1$. \instr{If} $i \leq R$, \instr{return} to Step~3, 
          \instr{otherwise halt}
\end{itemize}
}

\frame{\frametitle{Metropolis-Hastings sampler}
\begin{itemize}
 \item Because the accepted values may approach the correct distribution only slowly,
 we have to make sure the chain of samples has converged to a stationary sampling
 state before using the samples for inference\pause
 \item Specifically, we will discard a number of initial samples known as the \red{burn-in}:
 $$\hat\beta_0 = \frac{1}{R-B} \sum_{i=B}^R \beta_0^{(i)}$$ \pause
 \item We also repeat the procedure a few times with different values for $\theta^{(0)}$ to
 ensure that the algorithm converges to the same stationary distribution. Several convergence
 statistics exist, with Geweke's and Gelman's $\hat R$ being the most popular
\end{itemize}
}

\frame{\frametitle{Sampling in MATLAB}


\begin{exercise}[Metropolis sampling in MATLAB]
Write a new class called \texttt{ Metropolis() }

The constructor should work by taking an anonymous function as input (call the property \texttt{ Target } of type \texttt{ function\_handle }), with other inputs such as parameters of the algorithm optional (with defaults)

In addition to the standard methods, \texttt{ Metropolis() } should have at least a method \texttt{ DrawSamples(N) } to draw $N$ samples

You should assume that \texttt{ TargetLogPdf } has exactly one input, but the input may be a vector

Finally, make sure the \texttt{ disp() } method outputs something informative about the current state of the sampler

\end{exercise}
}


\end{document}



